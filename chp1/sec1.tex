\section{Data, what is it?}
In this section, we'll be working with data. Remember, data are \textbf{values}
along with their \textbf{context}.

\subsection{Context}
When describing the context of data, it is important to consider the W's
\begin{itemize}
  \item Who? - Who was the data collected from? 
  \item What? - The variables recorded from each experimental unit, and the unit of measurement.
What type is the variable? Quantitative or categorical?
  \item When? - When was the data collected?
  \item Where? - Where was the data collected?
  \item Why? - Why was the data collected?
  \item How? - How was the data collected?
\end{itemize}
Data without context is useless. For example, consider:
\begin{center}
  1.2, 3.7, 4.4, 2.2
\end{center}
Are these data? No. These are just numbers without any context. Numbers without any context mean nothing.

\subsection{Types of Data}
There are two types of data. These include \emph{categorical} and \emph{quantitative} data.
These two types of data must be expressed differently.

\begin{mdframed}
  \begin{definition}{\textbf{\underline{Categorical Variable:}}}
    A categorical variable is a variable that can take on one of a limited amount
  of possible values. Individuals are assigned to a group by some qualitative property.
  Examples: race, sex, eye color.
  \end{definition}
\end{mdframed}

\begin{mdframed}
  \begin{definition}{\textbf{\underline{Quantitative Variable:}}}
    A quantitative variable is a variable that is measured on a numeric scale.
    Examples: age, height, weight.
  \end{definition}
\end{mdframed}
